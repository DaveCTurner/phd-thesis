\documentclass[twoside]{article}
% vim: set foldlevel=0 cms=\%%s:

\usepackage{amsfonts}
\usepackage{amsmath}
\usepackage{amssymb}
\usepackage{amscd}
\usepackage{amsthm}
\usepackage{proof}
\usepackage{overlay}
\usepackage{graphicx}
\usepackage{color}
\usepackage{enumerate}
\usepackage[all]{xy}

\setlength\parskip{2ex}
\setlength\parindent{0pt}

\bibliographystyle{plain}

\input{macros.texinc}

\usepackage{fancyheadings}
\pagestyle{fancy}
\rfoot[]{\revisionnumber}
\lfoot[\revisionnumber]{}

\newtheorem{theorem}{Theorem}
\newtheorem{lemma}[theorem]{Lemma}
\newtheorem{corollary}[theorem]{Corollary}

\title{Nominal Domain Theory for Concurrency}
\date{\revisionnumber}
\author{David Turner}


\begin{document}
%\maketitle

%\input{definitions.texinc}

\section{Nominal HOPLA}

\subsection{Introduction}

Nominal HOPLA is an expressive calculus for higher-order processes with
nondeterminism and name-binding which will now be used to illustrate the
domain theory of the previous section. Nominal HOPLA follows closely the
development of HOPLA (Higher-Order Process LAnguage) \cite{nygaardwinskel1}
and is inspired by the language New-HOPLA \cite{zappanardelliwinskel}.

Recall from section \ref{??} that it is possible to add name-binding to set
theory by the use of presheaves or sheaves over $\mathbb{I}$, and that the
sheaf approach gives rise to the theory of nominal sets. Roughly speaking,
HOPLA, New-HOPLA and Nominal HOPLA are related to each other in the same way
that respectively set theory, presheaves over $\mathbb{I}$ and sheaves over
$\mathbb{I}$ are related: HOPLA has no names, New-HOPLA treats names
explicitly in its syntax, whereas the treatment of names in Nominal HOPLA is
much more implicit.

In order to develop (TODO wrong word) Nominal HOPLA it is necessary to give
the language an abstract syntax, and this syntax will include some binding
operators such as $\abstract{x}{t}$ which binds free occurrences of the
variable $\termvar{x}$ in the term $t$. One of the earliest applications of
nominal sets\cite{??} was to formalise common informal arguments about syntax
with binding. For example, when performing a proof by induction over the
syntax of a language with binding operators, bound variables are commonly
assumed to be fresh, and this can be shown to be a valid induction principle
by representing the syntax within nominal sets, where variables are
represented by names.

However, the structure of interest here is not the syntax of Nominal HOPLA but
its semantics, and the binding of variables in its syntax is a distraction.
To avoid confusion the binding of variables will be treated in the usual
informal fashion: bound variables will always be distinct from the other
variables in scope, and substitution will silently avoid capturing free
variables. In particular, if $\termvar{x}$ is a variable and $\sigma$ is a
permutation of the set of atoms then $\sigma \cdot \termvar{x} = \termvar{x}$.

\input{syntax.texinc}

%\input{subst.texinc}


%\input{typeenv.texinc}

\input{tjudge.texinc}

\input{substlemma.texinc}

\input{opsem.texinc}

\vfill\pagebreak

\input{densem.texinc}

%\vfill\pagebreak

\input{substlemmaden.texinc}

%\input{beta.texinc}
%\input{redsem.texinc}

\input{soundness.texinc}
\input{adequacy.texinc}

\bibliography{thesis}

\end{document}

