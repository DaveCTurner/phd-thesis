\documentclass[twoside]{article}
% vim: set foldlevel=0 cms=\%%s:

\usepackage{amsfonts}
\usepackage{amsmath}
\usepackage{amssymb}
\usepackage{amscd}
\usepackage{amsthm}
\usepackage{proof}
\usepackage{overlay}
\usepackage{graphicx}
\usepackage{color}
\usepackage{enumerate}
\usepackage[all]{xy}

\setlength\parskip{2ex}
\setlength\parindent{0pt}

\input{macros.texinc}

\usepackage{fancyheadings}
\pagestyle{fancy}
\rfoot[]{\revisionnumber}
\lfoot[\revisionnumber]{}

\newtheorem{theorem}{Theorem}
\newtheorem{lemma}[theorem]{Lemma}
\newtheorem{corollary}[theorem]{Corollary}

\title{Nominal Domain Theory for Concurrency}
\date{\revisionnumber}
\author{David Turner}


\begin{document}
%\maketitle

%\input{definitions.texinc}

%\section{Nominal HOPLA}

%\input{syntax.texinc}

%\input{subst.texinc}


%\input{typeenv.texinc}

\input{tjudge.texinc}

\input{substlemma.texinc}

\input{opsem.texinc}

\vfill\pagebreak

\input{densem.texinc}

%\vfill\pagebreak

\input{substlemmaden.texinc}

%\input{beta.texinc}
%\input{redsem.texinc}

\input{soundness.texinc}
\input{adequacy.texinc}

\end{document}

\end

\vfill\pagebreak


\subsection{Soundness}

\begin{lemma}\label{betasound}The following statements hold.
\begin{enumerate}
\item$\sem{\tjudge{\Gamma}{\termsubst{t}{u}{x}}{\typeP}} =
\sem{\tjudge{\Gamma}{\apply{\abstract{x}{t}}{u}}{\typeP}}$.
\item$\sem{\tjudge{\Gamma}{\termsubst{t}{u}{x}}{\typeP}} =
\sem{\tjudge{\Gamma}{\newapply{\newabstract{x}{t}}{u}}{\typeP}}$ where $u
\freshfor t$.
\item$\sem{\tjudge{\Gamma}{\termsubst{t}{u'}{x}}{\typeP}} \sqsubseteq
\sem{\tjudge{\Gamma}{\prefmatch{u}{x}{t}}{\typeP}}$ where $\sem{\bang{u'}}
\sqsubseteq \sem{u}$, with equality if $\sem{\bang{u'}} = \sem{u}$.
\item$\sem{\tjudge{\Gamma}{t}{\typePsub{\ell}}} =
\sem{\tjudge{\Gamma}{\lproj{\ell}{(\linj{\ell}{t}})}{\typePsub{\ell}}}$ where 
$\ell \in L$.\end{enumerate}
\end{lemma}

\begin{proof}
\begin{enumerate}
\item
\begin{eqnarray*}\sem{\apply{\abstract{x}{t}}{u}} &=& \mathrm{app}^\otimes
\circ (\mathrm{abs}^\otimes \sem{t} \otimes \sem{u}) \\ &=&
\mathrm{app}^\otimes \circ (\mathrm{abs}^\otimes \sem{t} \otimes
\mathbf{1}_{\mathbb P}) \circ (\mathbf{1}_{\Gamma} \otimes \sem{u})\\ &=&
\sem{t} \circ (\mathbf{1}_\Gamma \otimes \sem{u}) = \sem{\termsubst{t}{u}{x}}.
\end{eqnarray*}
\item
\begin{eqnarray*}\sem{\newapply{\newabstract{x}{t}}{u}} &=& \mathrm{app}^\boxtimes
\circ (\mathrm{abs}^\boxtimes \sem{t} \boxtimes \sem{u}) \\ &=&
\mathrm{app}^\boxtimes \circ (\mathrm{abs}^\boxtimes \sem{t} \boxtimes
\mathbf{1}_{\mathbb P}) \circ (\mathbf{1}_{\Gamma} \boxtimes \sem{u})\\ &=&
\sem{t} \circ (\mathbf{1}_\Gamma \boxtimes \sem{u}) =
\sem{\termsubst{t}{u}{x}}.
\end{eqnarray*}
\item
\begin{eqnarray*}\sem{\prefmatch{u}{x}{t}}
&=& \epsilon_{\mathbb P} \circ \sem{t}_\bot \circ s_{\Gamma, \mathbb
Q} \circ \Gamma[\sem{u}/x] \\
&\sqsupseteq& \epsilon_{\mathbb P} \circ \sem{t}_\bot \circ
s_{\Gamma, \mathbb Q} \circ \Gamma[\sem{\bang{u'}}/x] \\
&=& \epsilon_{\mathbb P} \circ \sem{t}_\bot \circ s_{\Gamma, \mathbb
Q} \circ \Gamma[\eta_{\mathbb Q}/x] \circ
\Gamma[\sem{u'}/x] \\
&=& \epsilon_{\mathbb P} \circ \sem{t}_\bot \circ \eta_\Gamma \circ
\Gamma[\eta_{\mathbb Q}/x] \circ \Gamma[\sem{u'}/x] \\
&=& \epsilon_{\mathbb P} \circ
\eta_{\mathbb P} \circ \sem{t} \circ \Gamma[\sem{u'}/x] \\
&=& \sem{t} \circ \Gamma[\sem{u'}/x] \\
&=& \sem{\termsubst{t}{u'}{x}}.
\end{eqnarray*}
\item 
\begin{eqnarray*}
\sem{\lproj{\ell}{(\linj{\ell}{t}})}
&=& \pi_{\ell} \circ \mathrm{inj}_{\ell} \circ \sem{t} \\
&=& \sem{t} 
\end{eqnarray*}
\end{enumerate}
\end{proof}

\begin{theorem}[Soundness]
\label{soundnessactions}
If $\ajudge{\typelift{\typeP}}{t}{\bangaction}{t'}$ then \[\bang{t'}
\ctxle_{\typelift{\typeP}} t, \text{ and}\] 
\[\sem{\tjudge{}{\bang{t'}}{\typelift{\typeP}}} \sqsubseteq 
	\sem{\tjudge{}{t}{\typelift{\typeP}}}.\]
\end{theorem}
\renewcommand\thetheorem{\arabic{theorem}}

% Proof of soundnessactions {{{
\begin{proof} The only action that $\bang{t'}$ may perform is
$\ajudge{\typelift{\typeP}}{\bang{t'}}{\bangaction}{t'}$, so the first
conclusion is immediate. The remainder of the proof is by induction on the
derivation of the transition. The base case is immediate. Note that each
inductive step is of the form
\[\infer{\ajudge{\typeP}{C(t_2)}{\bangaction}{t'}}
{\ajudge{\typeP}{C(t_1)}{\bangaction}{t'}}.\] By monotonicity of composition,
it is sufficient to show that $\sem{t_2} \sqsupseteq \sem{t_1}$ in each case.
\begin{enumerate}
%
\item[${!}E$] By lemma \ref{betasound}, $\sem{\prefmatch{\bang{u}}{x}{t}} =
\sem{\termsubst{t}{u}{x}}$.
%
\item[$\lambda I$] By lemma \ref{betasound},
$\sem{\apply{\abstract{x}{t}}{u}} = \sem{\termsubst{t}{u}{x}}$.
%
\item[$\nu I$] By lemma \ref{betasound}, if $u \freshfor t$ then
$\sem{\newapply{\newabstract{x}{t}}{u}} = \sem{\termsubst{t}{u}{x}}$.
%
\item[$\mathord{\bigoplus} I$] By lemma \ref{betasound},
$\sem{\lproj{\ell}{\linj{\ell}{t}}} = \sem{t}$.
%
\item[$\lambda E$] If $t = \apply{t_0}{u}$ then \[\sem{\termop{a}{t}} =
\sem{\termop{a}{\apply{t_0}{u} } } = \sem{\termop{(\mapaction{u}{a})}{t_0}}
\sqsupseteq \sem{\bang{t'}}\]
%
\item[$\nu I$] If $a = \newmapaction{u}{a_0}$ and $t = \newabstract{x}{t_0}$
then since $u \freshfor t$ \begin{eqnarray*}\sem{\termop{a}{t}} &=&
\sem{\termop{a_0}{\newapply{\newabstract{x}{t_0}}{u} }} =
\sem{\termop{a_0}{\termsubst{t_0}{u}{x} }} \sqsupseteq
\sem{\bang{t'}}\end{eqnarray*}
%
\item[$\nu E$] If $t = \newapply{t_0}{u}$ then \[\sem{\termop{a}{t}} =
\sem{\termop{a}{\newapply{t_0}{u} } } = \sem{\termop{(\newmapaction{u}{a})}{t_0}}
\sqsupseteq \sem{\bang{t'}}\]
%
\item[$\mathord{\bigoplus} I$] If $a = \labelaction{\ell_0}{a_0}$ and $t =
\linj{\ell_0}{t_0}$ then \begin{eqnarray*} \sem{\termop{a}{t}} =
\sem{\termop{a_0}{{t_0 } }} \sqsupseteq \sem{\bang{t'}} \end{eqnarray*}
%
\item[$\mathord{\bigoplus} E$] If $t = \lproj{\ell}{t_0}$ then
\[\sem{\termop{a}{t}} = \sem{\termop{a}{\lproj{\ell}{t_0} }} =
\sem{\termop{(\labelaction{\ell_0}{a})}{t_0}} \sqsupseteq \sem{\bang{t'}}\]
%
\item[$\Sigma I$]
If $t = \ndsum{i}{I}{t_i}$ then \begin{eqnarray*}
\sem{\termop{a}{t}} &=& \sem{\termop{a}{\ndsum{i}{I}{t_i} }} \\
&=& a^*\bigl(\textstyle\sum_{i \in I} \sem{t_i}\bigr) \sqsupseteq
a^*\bigl(\sem{t_{i_0}}\bigr) \sqsupseteq \sem{\bang{t'}} 
\end{eqnarray*}
%
\end{enumerate}
\end{proof}
% }}}

Define a relation $X \trianglelefteq_{\mathbb P} t$ between $X
\subseteq_\downarrow \mathbb P$ and closed terms of type $\mathbb P$. Define
also a relation $p \in_{\mathbb P} t$ between elements of $\mathbb P$ and
closed terms of type $\mathbb P$. $P$ is a set of paths with at most one
element.

% Definition of \trianglelefteq_{\mathbb P} {{{
\[\begin{array}{rcl}
X \trianglelefteq_{\mathbb P} t & \Longleftrightarrow & \forall p \in X.\ p
\in_{\mathbb P} t \\
p \in_{\typeO} t & \Longleftrightarrow & \bot \\
\alpha \linmapsto a;p \in_{\mathbb A \linmap \mathbb Q} t &
\Longleftrightarrow & a;p \in_{\mathbb Q} t(\alpha) \\
\alpha \newmapsto a;p \in_{\mathbb A \newmap \mathbb Q} t &
\Longleftrightarrow & a;p \in_{\mathbb Q} t[\alpha] \\
\ell_0{:}a;p \in_{\bigoplus_{\ell \in L} \mathbb P_\ell} t &
\Longleftrightarrow & a;p \in_{\mathbb P_{\ell_0}} \pi_{\ell_0} t\\
P \in_{\mathbb P_\bot} t & \Longleftrightarrow & \exists t'.\
\mathbb{P}_\bot : t \labelrightarrow{!} t' : \mathbb P \text{ and } P
\trianglelefteq_{\mathbb P} t'
\end{array}\]
% }}}

\begin{lemma} \label{triangleleftandsyntacticpreorder}
If $p \le_{\mathbb P} p'$ and $p' \in_{\mathbb P} t$ then $p \in_{\mathbb P}
t$.
\end{lemma}

% Proof of triangleleftandsyntacticpreorder {{{
\begin{proof}
The proof is by induction on the type $\mathbb P$.
\begin{itemize}
\item[$\typeO$] The hypotheses are never satisfied so the implication holds.
%
\item[$\mathbb P_\bot$] If $p' \in \mathbb P_\bot$ then either $p' = \{q'\}$
for some $q' \in \mathbb P$, or $p' = \varnothing$. If $p' = \varnothing$ then
$p = p'$ and the result follows. If $p' = \{q'\}$ then there exists a $t'$
such that $\mathbb P_\bot : t \labelrightarrow{!} t' : \mathbb P$ and $q'
\in_{\mathbb P} t'$. Either $p = \{q\}$ with $q \le_{\mathbb P} q'$ or $p =
\varnothing$. In the former case, by induction, $q \in_{\mathbb P} t'$ and
hence $p \in_{\mathbb P_\bot} t$, and in the latter case, it is clear that
$\varnothing \in_{\mathbb P_\bot} t$ as required.
%
\item[$\mathbb P \linmap \mathbb Q$] If $p' \in \mathbb P \linmap \mathbb Q$
then $p' = (\sem{u'}, q')$ for some $\vdash u : \mathbb P$ and some $q' \in
\mathbb Q$. Therefore $p = (\sem{u}, q)$ for some $q \le_{\mathbb Q} q'$ and
$u' \le_{\mathbb P} u$.  By definition $q' \in_{\mathbb Q} t(u')$ and so by
induction $q \in_{\mathbb Q} t(u)$, whence $p \in_{\mathbb P \linmap \mathbb
Q} t$ as required.
%
\item[$\mathbb P \newmap \mathbb Q$] If $p' \in \mathbb P \newmap \mathbb Q$
then $p' = [(\sem{u}, q')]_\sim$ for some $\vdash u : \mathbb P$ and some $q'
\in \mathbb Q$. Choosing a sufficiently fresh $u$, $p = [(\sem{u}, q)]_\sim$
for some $q \le_{\mathbb Q} q'$. By definition $q' \in_{\mathbb Q} t[u]$ and
so by induction $q \in_{\mathbb Q} t[u]$, whence $p \in_{\mathbb P \newmap
\mathbb Q} t$ as required.
%
\item[$\bigoplus_{\ell \in L} \mathbb P_\ell$] If $p' \in \bigoplus_{\ell \in
L} \mathbb P_\ell$ then $p' = \ell_0{:}q'$ for some $\ell_0 \in L$ and $q' \in
\mathbb P_{\ell_0}$. Therefore $p = \ell_0{:}q$ for some $q \in \mathbb
P_{\ell_0}$, so that $q' \in_{\mathbb P_{\ell_0}} \pi_{\ell_0} t$ implies $q
\in_{\mathbb P_{\ell_0}} \pi_{\ell_0} t$ and the result follows.
%
\end{itemize}
\end{proof}
% }}}

\begin{lemma}\label{ctxle1}
If $p \in_{\mathbb P} t_1$ and $t_1 \ctxle_{\mathbb P} t_2$ then $p
\in_{\mathbb P} t_2$.
\end{lemma}

% Proof of ctxle1 {{{
\begin{proof}
The proof is by induction on the type $\mathbb P$.
\begin{itemize}
%
\item[$\typeO$] The hypotheses are never satisfied so the implication holds.
%
\item[$\mathbb P_\bot$] If $p \in_{\mathbb P_\bot} t_1$
then there exists $t'$ such that $\mathbb P_\bot : t_1 \labelrightarrow{!}
t'$, and hence $\mathbb P_\bot : t_2 \labelrightarrow{!} t'$. If $p = \{q\}$
where $q \in \mathbb P$ then furthermore $q \in_{\mathbb P} t'$; in any case
$p \in_{\mathbb P_\bot} t_2$ as required.
%
\item[$\mathbb P \linmap \mathbb Q$] If $p \in \mathbb P \linmap \mathbb Q$
then $p = (\sem{u}, q)$ for some $\vdash u : \mathbb P$ and $q \in \mathbb Q$,
and hence $q \in_{\mathbb Q} t(u)$. But $t_1(u) \ctxle_{\mathbb Q} t_2(u)$ and
by induction $q \in_{\mathbb P} t_2(u)$. Therefore $p \in_{\mathbb P \linmap
\mathbb Q} t_2$, as required.
%
\item[$\mathbb P \newmap \mathbb Q$] If $p \in \mathbb P \newmap \mathbb Q$
then $p = [(\sem{u}, q)]_\sim$ for some $\vdash u : \mathbb P$ and $q \in
\mathbb Q$, and hence $q \in_{\mathbb Q} t[u]$. But similarly to the previous
case, $t_1[u] \ctxle_{\mathbb Q} t_2[u]$ and by induction $q \in_{\mathbb Q}
t_2[u]$. Therefore $p \in_{\mathbb P \newmap \mathbb Q} t_2$, as required.
%
\item[$\bigoplus_{\ell \in L} \mathbb P_\ell$] If $p \in \bigoplus_{\ell \in
L} \mathbb P_\ell$ then $p = \ell_0{:}q$ for some $\ell_0 \in L$ and some $q
\in \mathbb P_{\ell_0}$ and hence $q \in_{\mathbb P_{\ell_0}} \pi_{\ell_0} t_1$.
But again $\pi_{\ell_0} t_1 \ctxle_1 \pi_{\ell_0} t_2$ so that $q \in_{\mathbb
P_{\ell_0}} \pi_{\ell_0} t_2$. Therefore $p \in_{\bigoplus_{\ell \in L}
\mathbb P_\ell} t_2$ as required.
%
\end{itemize}
\end{proof}
% }}}

\begin{lemma}\label{triangleleftanddenotations}
If $p \in_{\mathbb P} t$ then $p \in \sem{\vdash t : \mathbb P}$.
\end{lemma}

% Proof of triangleleftanddenotations {{{
\begin{proof}
The proof is by induction on the type $\mathbb P$.
\begin{itemize}
%
\item[$\typeO$] The hypotheses are never satisfied so the implication holds.
%
\item[$\mathbb P_\bot$] If $p \in_{\mathbb P_\bot} t$ then there is a $t'$
such that $\mathbb P_\bot : t \labelrightarrow{!} t'$ and $p
\trianglelefteq_{\mathbb P} t'$. Hence for all $q \in p$, $q \in_{\mathbb P}
t'$, and so by induction, $q \in \sem{t'}$ and therefore $p \in \sem{!t'}$.
But by soundness, $\sem{!t'} \subseteq \sem{t}$ and hence $p \in \sem{t}$ as
required.
%
\item[$\mathbb P \linmap \mathbb Q$] If $p_0 \in_{\mathbb P \linmap \mathbb
Q} t$ then $p_0 = (\sem{u}, p)$ and $p \in_{\mathbb Q} t(u)$. Hence by
induction $p \in \sem{t(u)}$ so that $p_0 = (\sem{u}, p) \in \sem{t}$ as
required.
%
\item[$\mathbb P \newmap \mathbb Q$] If $p_0 \in_{\mathbb P \newmap \mathbb
Q} t$ then $p_0 = [(\sem{u}, p)]_\sim$ and $p \in_{\mathbb Q} t[u]$. Hence
by induction $p \in \sem{t[u]}$ so that $p_0 = [(\sem{u}, p)]_\sim \in
\sem{t}$ as required.
%
\item[$\bigoplus_{\ell \in L} \mathbb P_\ell$] If $p_0 \in_{\bigoplus_{\ell
\in L} \mathbb P_\ell} t$ then $p_0 = \ell_0{:}p$ for some $\ell_0 \in L$ and
$p \in_{\mathbb P_{\ell_0}} \pi_{\ell_0} t$. Hence by induction $p \in
\sem{\pi_{\ell_0} t}$ so that $\ell_0{:}p \in \sem{t}$ as required.
%
\end{itemize}
\end{proof}
% }}}

\begin{lemma}\label{biglemma}
Suppose $\Gamma \vdash t : \mathbb P$, let $\gamma \in \Gamma$ and define
$\gamma_x = \sem{\Gamma \vdash x : \mathbb P_x}(\gamma)$ to be the
$x$-component of $\gamma$ for each $x \in \dom{\Gamma}$. Also let $s_x$ be a
closed term such that $\vdash s_x : \mathbb P_x$ and $\gamma_x
\trianglelefteq_{\mathbb P_x} s_x$ for each $x \in \dom{\Gamma}$. Then
\[\sem{\Gamma \vdash t : \mathbb P}(\gamma) \trianglelefteq_{\mathbb P}
t[s_x/x]_{x \in \dom{\Gamma}}\] where $t[s_x/x]_{x \in \dom{\Gamma}}$ is the
term obtained by substituting each $x \in \dom{\Gamma}$ with $s_x$.
\end{lemma}

% Proof of biglemma {{{
\begin{proof}
The proof is by induction on the typing judgement $\Gamma \vdash t : \mathbb
P$.
%
\begin{itemize}
%
\item[$\otimes C$] For each contracted variable pair, let $x^+$ be the
variable remaining after the contraction and $x^-$ be the lost variable. Apply
the induction hypothesis with $s_{x^-} = s_{x^+}$.
%
\item[$\otimes / \boxtimes$] If $\gamma \in \Gamma[\Delta \boxtimes
\Delta'/x]$ then $\gamma \in \Gamma[\Delta \otimes \Delta'/x]$. Apply the
induction hypothesis.
%
\item[$\otimes W$] The weakened term does not mention any of the new
variables, so the induction hypothesis applies.
%
\item[$\boxtimes W$] Similarly.
%
\item[$\mathit{var}$] If $\Gamma = x : \mathbb P$ and $t = x$ then for any
$\gamma \in \Gamma$, $\gamma_x = \gamma$ and $t[s_x/x] = s_x$ so this case is
trivially true.
%
\item[$\mathrm{rec}I$] %TODO
%
\item[$!I$] The induction hypothesis states that $\sem{\Gamma \vdash t :
\mathbb P}(\gamma) \trianglelefteq_{\mathbb P} t[s_x/x]_{x \in \dom{\Gamma}}$,
so that $\sem{\Gamma \vdash t : \mathbb P}(\gamma) \in_{\mathbb P_\bot}
{!}t[s_x/x]_{x \in \dom{\Gamma}}$. But $\sem{\Gamma \vdash {!}t : \mathbb
P}(\gamma) = \{ \sem{\Gamma \vdash t : \mathbb P}(\gamma)\}$ and the result
follows.
%
\item[$!E$] As a useful shorthand, write $u^{[s]}$ for $u[s_x/x]_{x \in
\dom\Lambda}$. The induction hypothesis says that for each $q \in \sem{u}
(\gamma_u)$, $q \in_{\mathbb Q_\bot} u^{[s]}$. Therefore there is a $u_0$ such
that $\mathbb Q_{\bot}: u^{[s]} \labelrightarrow{!} u_0$ and $q
\trianglelefteq_{\mathbb Q} u_0$, and hence \[\textstyle q \in_{\mathbb Q_\bot} {!}u_0
\ctxle_{\mathbb Q_\bot} \sum_{u^{[s]} \labelrightarrow{!} u'} {!}u' \quad \text{and hence}
\quad q \in \bigsem{\sum_{u^{[s]} \labelrightarrow{!} u'} {!}u'}\] Thus
\[\textstyle\sem{u}(\gamma_u) \subseteq \bigsem{\sum_{u^{[s]}
\labelrightarrow{!} u'} {!}u'} = \sum_{u^{[s]} \labelrightarrow{!} u'}
\eta_{\mathbb Q} \circ \sem{u'}\]
and hence
\begin{eqnarray*}
\sem{[u > {!}x \Rightarrow t]} (\gamma) & = & \epsilon_{\mathbb P} \circ
\sem{t}_\bot \circ s_{\Gamma, x_0} \circ \Gamma[\sem{u}/x_0] (\gamma) \\ &
\subseteq & \textstyle
\sum_{u^{[s]} \labelrightarrow{!} u'} \epsilon_{\mathbb P} \circ \sem{t}_\bot
\circ s_{\Gamma, x_0} \circ \Gamma[ \eta_{\mathbb Q} \circ \sem{u'}/x_0] \\
&=&\textstyle
\sum_{u^{[s]} \labelrightarrow{!} u'} \epsilon_{\mathbb P} \circ \sem{t}_\bot
\circ \eta_{\Gamma[\mathbb Q/x_0]} \circ \Gamma[ \sem{u'}/x_0] \\
&=&\textstyle
\sum_{u^{[s]} \labelrightarrow{!} u'} \epsilon_{\mathbb P} \circ \eta_{\mathbb
P} \circ \sem{t} \circ \Gamma[ \sem{u'}/x_0] \\
&=& \textstyle
\sum_{u^{[s]} \labelrightarrow{!} u'} \sem{t[u'/x]} = \bigsem{\sum_{u^{[s]}
\labelrightarrow{!} u'} t[u'/x]}\\
&\trianglelefteq_{\mathbb P} & \textstyle\sum_{u^{[s]} \labelrightarrow{!} u'}
t[u'/x] [s_x/x]_{x \in \dom\Gamma} \\
&\ctxle_{\mathbb P}& [u>{!}x \Rightarrow t][s_x/x]_{x \in \dom{\Gamma[\Lambda/x_0]}} 
\end{eqnarray*}
Where the last step relies on the observation (proven by induction on $\mathbb
P$) that $\sum_{u \labelrightarrow{!} u'} t[u'/x] \ctxle_{\mathbb P} [u > {!}x
\Rightarrow t]$.
%
\item[$\lambda I$] If $p_0 \in \sem{\Gamma \vdash \lambda x.t : \mathbb P
\linmap \mathbb Q}(\gamma)$ then by the denotational semantics there is
$\vdash u : \mathbb P$, $q \in \gamma$, and $p \in \sem{t}(\sem{u}, q)$ such
that $p_0 = (\sem{u}, p)$. By induction $\sem{u} \trianglelefteq_{\mathbb P}
u$ and also \[p \in_{\mathbb P} t[s_x/x]_{x \in
\dom\Gamma}[u/x] \ctxle_{\mathbb Q} \lambda x.t[s_x/x]_{x \in \dom\Gamma}
(u)\] and hence $(\sem{u}, p) \in_{\mathbb P \linmap \mathbb Q} \lambda
x.t[s_x/x]_{x \in \dom\Gamma}$ as required.
%
\item[$\lambda E$] If $p \in \sem{t(u)}(\gamma)$ then $(\sem{u}, p) \in
\sem{t}(\gamma)$. By induction, \[(\sem{u}, p) \in_{\mathbb P \linmap \mathbb
Q} t[s_x/x]_{x \in \dom\Gamma}\] and hence $p \in_{\mathbb Q}
t(u)[s_x/x]_{x \in \dom\Gamma}$ as required.
%
\item[$\nu I$] If $p_0 \in \sem{\Gamma \vdash \nu x.t : \mathbb P \newmap
\mathbb Q}(\gamma)$ then by the denotational semantics there is $\vdash u :
\mathbb P$ and $q \in \gamma$ with $\sem{u} \freshfor q$, and $p \in
\sem{t}(\sem{u}, q)$ such that $p_0 = [(\sem{u}, p)]_\sim$. By
finite-supportness, there is such an $u$ such that $u \freshfor t$ as well. By
induction, \[p \in_{\mathbb Q} t[s_x/x]_{x \in \dom\Gamma}[u/x]
\ctxle_{\mathbb Q} \nu x.t[s_x/x]_{x \in \dom\Gamma} [u]\] and hence
$[(\sem{u}, p)]_\sim \in_{\mathbb P \newmap \mathbb Q} \nu x.t[s_x/x]_{x \in
\dom\Gamma}$ as required.
%
\item[$\nu E$] If $p \in \sem{t[u]}(\gamma)$ then $[(\sem{u}, p)]_\sim \in
\sem{t}(\gamma)$. By induction, \[[(\sem{u}, p)]_\sim \in_{\mathbb P \newmap
\mathbb Q} t[s_x/x]_{x \in \dom\Gamma}\] and hence $p \in_{\mathbb Q}
t[u][s_x/x]_{x \in \dom\Gamma}$ as required.
%
\item[$\mathord{\bigoplus} I$] If $p_0 \in \sem{\linj{\ell_0}{t}} (\gamma)$
then there is  $p \in \sem{t}(\gamma)$ such that $p_0 = \ell_0{:}p$. By
induction, \[p \in_{\mathbb P_{\ell_0}} t[s_x/x]_{x \in \dom\Gamma}
\ctxle_{\mathbb P_{\ell_0}} \lproj{\ell_0}{(\linj{\ell_0}{t})}
 [s_x/x]_{x \in \dom\Gamma}\] so that $\ell_0{:}p \in_{\bigoplus_{\ell
\in L} \mathbb P_{\ell}} \linj{\ell_0}{t} [s_x/x]_{x \in \dom\Gamma}$ as required.
%
\item[$\mathord{\bigoplus} E$] If $p \in \sem{\pi_{\ell_0} t}(\gamma)$ then
$\ell_0{:}p \in \sem{t}(\gamma)$. By induction, \[\ell_0{:}p
\in_{\bigoplus_{\ell \in L} \mathbb P_\ell} t[s_x/x]_{x \in \dom\Gamma}\] and
hence $p \in_{\mathbb P_{\ell_0}} \pi_{\ell_0} t[s_x/x]_{x \in \dom\Gamma}$ as
required.
%
\item[$\Sigma I$] If $p \in \sem{\sum_{i \in I} t_i} (\gamma)$ then there is
an $i_0 \in I$ such that $p \in \sem{t_{i_0}} (\gamma)$. By induction, \[p
\in_{\mathbb P} t_{i_0}[s_x/x]_{x\in\dom\Gamma} \ctxle_{\mathbb P}
\textstyle\sum_{i \in I} t_i[s_x/x]_{x\in\dom\Gamma}\] as required.
%
\end{itemize}
\end{proof}
% }}}

\begin{lemma}[Main Lemma]
\label{mainlemma}
Suppose $\vdash t : \mathbb P$. Then $\sem{t} \trianglelefteq_{\mathbb P} t$.
\end{lemma}

% Proof of mainlemma {{{
\begin{proof}
Let $\Gamma = \mathbb I$ and apply lemma \ref{biglemma}.
\end{proof}
% }}}

\begin{theorem}[Adequacy]\label{adequacy}
$\sem{\vdash t : \mathbb P_\bot} = \sem{\vdash \varnothing : \mathbb P_\bot}$
if and only if there exists no $t'$ such that $\mathbb P_\bot :  t
\labelrightarrow{!} t'$.
\end{theorem}

% Proof of adequacy {{{
\begin{proof}
Suppose that there exists a $t'$ such that $\mathbb P_\bot : t
\labelrightarrow{!} t'$. Then $\vdash t' : \mathbb P$, and by soundness
$\sem{!t'} \sqsubseteq \sem{t}$.  But $\sem{!t'} = \eta_{\mathbb P} \circ
\sem{t'}$ and $\eta_{\mathbb P}(X) \ne \varnothing$ for any $X
\subseteq_\downarrow \mathbb P$ and therefore $\sem{t} \ne \sem{\varnothing}$.

Conversely, suppose $\sem{t} \ne \varnothing$. Then since $\sem{t}
\subseteq_\downarrow \mathbb P_\bot$, $\varnothing \in \sem{t}$. Hence by
lemma \ref{mainlemma}, $\varnothing \in_{\mathbb P_\bot} t$ and so there is a
$t'$ such that $\mathbb P_\bot : t \labelrightarrow{!} t'$ as required.
\end{proof}
% }}}

\section{Full Abstraction} %{{{

Paths are given by the grammar \[p ::= \emptypath \mid \bangpath{p} \mid
\labelpath{\ell}{p} \mid \mappath{u}{p} \mid \newmappath{u}{p} \] where
$\ell$ is a label and $u$ is a closed term.



%
%Say that $t$ is a closed term if $\mathbb O \vdash t : \mathbb P$ where
%$\mathbb O$ is the empty environment. Say that $t$ is a program if it is a
%closed term of type $\mathbb O_\bot$.
%
%Let there be a distinguished variable, denoted $(-)$, and say that a term $C$
%is a $(\Lambda, \mathbb P)$ program context if, whenever $\Lambda \vdash t :
%\mathbb P$, $C[t/(-)]$ is a program. As a convenient shorthand, write $C(t)$
%for $C[t/(-)]$.
%
%Define the contextual preorder $\ctxle$ by the following. Let $t_1$ and
%$t_2$ be terms such that $\Lambda \vdash t_i : \mathbb P$ for $i = 1, 2$. Then
%$t_1 \ctxle t_2$ if and only if for all $(\Lambda, \mathbb P)$ program
%contexts it is the case that $\sem{C(t_1)} \ne \sem{\varnothing} \Rightarrow
%\sem{C(t_2)} \ne \sem{\varnothing}$. Notice that, in the light of the adequacy
%theorem above, $t_1 \ctxle t_2$ if and only if for all $C$, $C(t_2)$ can
%perform a ${!}$ action whenever $C(t_1)$ can.
%
%It will be necessary to consider finite strings $p$ of actions. The empty
%string is written $\varnothing$ and pairs of strings are concatenated as $p_1;
%p_2$. Actions may be assigned types as follows.
%
%\[\begin{array}{ccc}
%\infer{\varnothing : \mathbb P}{-} &
%\infer{{!}{;}p : \mathbb P_\bot}{p : \mathbb P} &
%\infer[\ell_0 \in L]{\ell_0{:}a{;}p : \bigoplus_{\ell \in L} \mathbb
%P_\ell}{a{;}p : \mathbb P_{\ell_0}} \\&
%\rule{0pt}{7ex}
%\infer{\alpha \linmapsto a{;}p : \mathbb A \linmap \mathbb P}{a{;}p : \mathbb
%P} &
%\infer{\alpha \newmapsto a{;}p : \mathbb A \newmap \mathbb P}{a{;}p : \mathbb
%P}
%\end{array}\]
%
%The typing judgement $p : \mathbb P$ may be interpreted as a linear map
%$\mathbb I \to \mathbb P$ as follows.
%
%\[\begin{array}{ccc}
%\sem{\varnothing : \mathbb P} = \bot &
%\sem{{!}{;}p : \mathbb P_\bot} = \eta_{\mathbb P} \circ \sem{p : \mathbb P} &
%\sem{\ell_0{:}a{;}p : \bigoplus_{\ell \in L} \mathbb
%P_\ell} = \mathrm{inj}_{\ell_0} \circ \sem{a{;}p : \mathbb P_{\ell_0}} \\
%\rule{0pt}{5ex} &
%\multicolumn{2}{c}{ \sem{\alpha \linmapsto a{;}p : \mathbb A \linmap \mathbb
%P} = \mathrm{abs}^\otimes ({=}\alpha{?} \circ \sem{a{;}p : \mathbb P})} \\
%\rule{0pt}{5ex} &
%\multicolumn{2}{c}{ \sem{\alpha \newmapsto a{;}p : \mathbb A \newmap \mathbb
%P} = \mathrm{abs}^\boxtimes ({=}\alpha{?} \circ \sem{a{;}p : \mathbb P})} \\
%\end{array}\]
%where ${=}\alpha{?}$ is the linear map $\mathbb A \to \mathbb I$ which sends
%$X \subseteq_\downarrow \mathbb A$ to $\varnothing$ if and only if $\alpha
%\notin X$. 
%
%More informally, the typing judgement $p : \mathbb P$ may be
%interpreted as an element $\sem{p : \mathbb P} \in \mathbb P$ as follows.
%\[\begin{array}{ccc}
%\sem{\varnothing : \mathbb P} = \bot &
%\sem{{!}{;}p : \mathbb P_\bot} = \{\sem{p : \mathbb P}\} &
%\sem{\ell_0{:}a{;}p : \bigoplus_{\ell \in L} \mathbb
%P_\ell} = \mathrm{inj}_{\ell_0}(\sem{a{;}p : \mathbb P_{\ell_0}}) \\
%\rule{0pt}{5ex} &
%\multicolumn{2}{c}{ \sem{\alpha \linmapsto a{;}p : \mathbb A \linmap \mathbb
%P} = (\alpha, \sem{a{;}p : \mathbb P})} \\
%\rule{0pt}{5ex} &
%\multicolumn{2}{c}{ \sem{\alpha \newmapsto a{;}p : \mathbb A \newmap \mathbb
%P} = [(\alpha, \sem{a{;}p : \mathbb P})]_{\sim}}
%\end{array}\]
%
%\begin{theorem}[Full Abstraction] Suppose that $t_1$ and $t_2$ are terms such
%that $\Lambda \vdash t_i : \mathbb P$ for $i = 1, 2$. Then \[\sem{t_1}
%\sqsubseteq \sem{t_2} \quad \Longleftrightarrow \quad t_1 \ctxle t_2\]
%\end{theorem}
%
%\begin{proof} Firstly, suppose that $\sem{t_1} \sqsubseteq \sem{t_2}$ and let
%$C$ be a $(\Lambda, \mathbb P)$ program context with $\sem{C(t_2)} =
%\sem{\varnothing}$. Then by the substitution lemma and the monotonicity of
%composition, $\sem{C(t_1)} \sqsubseteq \sem{C(t_2)} = \sem{\varnothing}$ and
%hence $t_1 \ctxle t_2$ as required.
%
%For the convese implication, it is necessary to consider finite strings $p$ of
%actions. The empty string is written $\varnothing$ and pairs of strings are
%concatenated as $p_1; p_2$. Define a program context $C_p$ which tests for a
%particular string $p$ by induction as follows.
%
%\[\begin{array}{c}
%C_\varnothing = {!}\varnothing \\
%C_{{!}; p} = [(-) > {!}x \Rightarrow C_p(x)] \\
%C_{\ell{:} a; p} = C_{a; p}(\pi_\ell (-))\\
%C_{\alpha \linmapsto a; p} = C_{a; p}((-)(\alpha))\\
%C_{\alpha \newmapsto a; p} = C_{a; p}((-)[\alpha])
%\end{array}
%\]
%
%%TODO So strings of actions are prime elements in the denotations. Go into
%%this.
%
%%TODO prove this next statement
%%TODO What does \in \sem{t} really mean? \sem{t} is a morphism.
%Note that if $p \in \sem{t}$ and there is a prime below $p$ which corresponds
%to the string $p_0$ of actions then $C_p(t)$ is a well-typed program; the
%interesting case is when $p$ contains actions which are constructed using
%$\newmapsto$.
%
%
%
%
%
%\end{proof}
%
%
%
%
%
%
%
% }}}
\end{document}
