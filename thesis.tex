\documentclass[twoside]{article}
% vim: set foldlevel=0 cms=\%%s:

\usepackage{amsfonts}
\usepackage{amsmath}
\usepackage{amssymb}
\usepackage{amscd}
\usepackage{amsthm}
\usepackage{proof}
\usepackage{overlay}
\usepackage{graphicx}
\usepackage{color}
\usepackage{enumerate}
\usepackage[all]{xy}

\setlength\parskip{2ex}
\setlength\parindent{0pt}

\bibliographystyle{plain}

\input{macros.texinc}

\usepackage{fancyheadings}
\pagestyle{fancy}
\rfoot[]{\revisionnumber}
\lfoot[\revisionnumber]{}

\newtheorem{theorem}{Theorem}
\newtheorem{lemma}[theorem]{Lemma}
\newtheorem{corollary}[theorem]{Corollary}

\title{Nominal Domain Theory for Concurrency}
\date{\revisionnumber}
\author{David Turner}


\begin{document}
%\maketitle

%\input{definitions.texinc}

\section{Nominal HOPLA}

\subsection{Introduction}

Nominal HOPLA is an expressive calculus for higher-order processes with
nondeterminism and name-binding which will now be used to illustrate the
domain theory of the previous section. Nominal HOPLA follows closely the
development of HOPLA (Higher-Order Process LAnguage) \cite{nygaardwinskel1}
and is inspired by the language New-HOPLA \cite{zappanardelliwinskel}.

Recall from section \ref{??} that it is possible to add name-binding to set
theory by the use of presheaves or sheaves over $\mathbb{I}$, and that the
sheaf approach gives rise to the theory of nominal sets. Roughly speaking,
HOPLA, New-HOPLA and Nominal HOPLA are related to each other in the same way
that respectively set theory, presheaves over $\mathbb{I}$ and sheaves over $\mathbb{I}$
are related.


\input{syntax.texinc}

%\input{subst.texinc}


%\input{typeenv.texinc}

\input{tjudge.texinc}

\input{substlemma.texinc}

\input{opsem.texinc}

\vfill\pagebreak

\input{densem.texinc}

%\vfill\pagebreak

\input{substlemmaden.texinc}

%\input{beta.texinc}
%\input{redsem.texinc}

\input{soundness.texinc}
\input{adequacy.texinc}

\bibliography{thesis}

\end{document}

